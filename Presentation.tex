\documentclass{beamer}

\usepackage[utf8]{inputenc}
\usepackage[french]{babel}
\usepackage[T1]{fontenc}
\usepackage{lmodern}
\usepackage{tikz}
\usepackage{multicol}

\newcommand*{\escape}[1]{\texttt{\textbackslash#1}}

\usetheme{PaloAlto}

\title{Protocole RINGO}
\subtitle{Rapport du Projet}
\author[]{ELBEZ Samuel 21200353, \\ JACQUETTE Pierrick 21305551}
\date{Avril 2016}
\institute[L3 S6 -- Informatique]{Université Paris 7 Diderot}

\begin{document}

\begin{frame}
	\titlepage
\end{frame}

\begin{frame}
	\frametitle{Sommaire}
	\tableofcontents	
\end{frame}

\section{Entité}
\begin{frame}
	\begin{center}
		{\Huge Une entité}
	\end{center}
\end{frame}
\begin{frame}{Une entité}
	 \textbf{Les caractéristiques d'une entité}
	 \begin{itemize}
		 \item<2-12> Un identifiant (8 caractères et unique)
 		 \item<3-12> Une chaîne de caractères pour le port In UDP (< 9999)
		 \item<4-12> Un tableau de chaîne de caractères pour le port Out UDP (< 9999)
		 \item<5-12> Un tableau de chaîne de caractères pour l'adresse Out
		 \item<6-12> Une chaîne de caractères pour le port TCP In
		 \item<7-12> Une chaîne de caractères pour le port TCP Out
		 \item<8-12> Un tableau de chaîne de caractères pour l'adresse IPV4 multi-diffusion (panne du réseau)
		 \item<9-12> Un tableau de chaîne de caractères pour le port de multi-diffusion (< 9999) 
		 \item<10-12> Un boolean pour savoir si c'est un duplicateur
		 \item<11-12> Une liste de chaîne de caractères de l'identifiant du message transmis
		 \item<12-12> Une liste de chaîne de caractères du nom d'un fichier
	\end{itemize}
\end{frame}

\section{Ports}
\begin{frame}
	\begin{center}
		{\Huge L'utilisation des différents ports}
	\end{center}
\end{frame}
\begin{frame}{L'utilisation des différents ports}
	 \begin{itemize}
		 \item<1-2> \textbf{UDP} écoute et transmet les messages
 		 \item<2-2> \textbf{TCP} permet l'insertion ou duplication d'un anneau
	\end{itemize}
\end{frame}


\section{Conventions}
\begin{frame}
	\begin{center}
		{\Huge Les conventions}
	\end{center}
\end{frame}

\begin{frame}{Les conventions}
	 \begin{itemize}
		 \item<1-5> La syntaxe des messages
		 \item<2-5> Un message a une taille maximale de 512 octets
 		 \item<3-5> Une entité recevant un message déjà transmis, on ne le retransmet pas
 		 \item<4-5> Si on reçoit un message mal formé, on ne le retransmet pas
 		 \item<5-5> Une entité reçoit un message d'application si elle supporte déclenche une action sinon transmet
	\end{itemize}
\end{frame}

\section{Spécification}
\begin{frame}
	\begin{center}
		{\Huge La spécification des messages}
	\end{center}
\end{frame}

\begin{frame}{La spécification des messages}
	 \begin{itemize}
		 \item<1-11> Les 4 premiers octets : \textbf{WELC, NEWC, ACKC, APPL, WHOS, MEMB, GBYE, EYBG, TEST, DOWN, DUPL, ACKD}
		 \item<2-11> \textbf{ip, ip-diff et ip-succ :} 15 oct, chaîne de caractères de l'adresse IPV4
 		 \item<3-11> \textbf{port, port-diff et port-succ :} 4 oct, chaîne de caractères du numéro de port
 		 \item<4-11> \textbf{idm et id-trans :} 8 oct unique
 		 \item<5-11> \textbf{id-app :} 8 oct, chaîne de caractères de l'identité de l'app
 		 \item<6-11> \textbf{id :} 8 oct, chaîne de caractères
 		 \item<7-11> \textbf{size-mess :} 3 oct, chaîne de caractères 
 		 \item<8-11> \textbf{size-nom :} 2 oct, chaîne de caractères
 		 \item<9-11> \textbf{num-mess et no-mess :} 8 oct, little endian
 		 \item<10-11> \textbf{size-content :} 3 oct, chaîne de caractères
	\end{itemize}
	\visible<11-11>{
		size-mess, size-nom et size-content complété par des 0 au début
	}
\end{frame}

\section{Messages}
\begin{frame}
	\begin{center}
		{\Huge Les différents types de messages}
	\end{center}
\end{frame}

\begin{frame}{Les différents types de messages}
	\textbf{Les messages du protocole : }
	\begin{itemize}
		 \item<2-9> \textbf{Insertion :} WELC ip port ip-diff port-diff \escape{n}
 		 \item<3-9> \textbf{Qui :} WHOS idm
		 \item<4-9> \textbf{Suppression entité :} GBYE idm ip port ip-succ port-succ
		 \item<5-9> \textbf{Tester un anneau :} TEST idm ip-diff port-diff
		 \item<6-9> \textbf{Duplication :}  DUPL ip port ip-diff port-diff \escape{n}
		 \item<7-9> \textbf{Suppression des messages :}  SUPP idm idmA
	\end{itemize}
	\textbf{Les messages d'application : }
	\begin{itemize}
 		 \item<8-9> \textbf{Diffusion de messages :} APPL idm DIFF\#\#\#\# size-mess mess
 		 \item<9-9> \textbf{Transfert de fichiers :} APPL idm TRANS\#\#\# REQ size-nom nom-fichier
	\end{itemize}
	
\end{frame}

\subsection{Messages protocolaires}
\begin{frame}
	\begin{center}
		{\Huge Les messages protocolaires}
	\end{center}
\end{frame}

\subsubsection{Insertion}
\begin{frame}
	\begin{center}
		{\Huge L'insertion}
	\end{center}
\end{frame}
\begin{frame}{L'insertion}
	\begin{tikzpicture}
			\node[draw] (E1) at (-3,0) {Entité 1};
			\node[draw] (E2) at (3,0) {Entité 2};
			\node[draw] (N) at (0,1) {Nouveau};
			\node[draw=none] (d1) at (0,-0.1) {};
			\visible<1>{
				\draw[->,>=latex] (N) -- (d1);
			}
			\visible<1-5>{
				\draw[->,>=latex] (E1) -- (E2);
			}
			\visible<2>{
				\draw[->,>=latex] (N) to[bend right] node[sloped,midway,above]{connect} (E1);
			}
			\visible<3>{
				\draw[->,>=latex] (E1) to[bend left] node[sloped,midway,above]{WELC} (N);
			}
			\visible<4>{
				\draw[->,>=latex] (N) to[bend right] node[sloped,midway,above]{NEWC} (E1);
			}
			\visible<5>{
				\draw[->,>=latex] (E1) to[bend left] node[sloped,midway,above]{ACKC} (N);
			}
			\visible<6>{
				\draw[->,>=latex] (E1) to[bend left](N);
				\draw[->,>=latex] (N) to[bend left] (E2);
			}
			\draw[->,>=latex] (E2) to[bend left] (E1);

	\end{tikzpicture}
	\begin{itemize}
		\item<2-6> Nouveau se connecte à Entité 1
		\item<3-6> Entité 1 envoi \textbf{WELC ip port ip-diff port-diff \escape{n}} (adresse et port de l’entité 2, adresse et port de multi-diffusion)
		\item<3-6> Nouveau reçoit WELC ip port ip-diff port-diff \escape{n}, stocke
		\item<4-6> Nouveau envoi \textbf{NEWC ip port \escape{n}} (information qu'il connaît dans ses attributs : adresse et port de Nouveau)
		\item<4-6> Entité 1 reçoit NEWC ip port \escape{n}, stocke
		\item<5-6> Entité 1 envoi \textbf{ACKC \escape{n}}
		\item<6-6> Entité 1 ferme la connexion
		\item<5-6> Nouveau reçoit ACKC \escape{n} et ne le transfert pas
	\end{itemize}
\end{frame}

\subsubsection{Qui est là ?}
\begin{frame}
	\begin{center}
		{\Huge Qui est là ?}
	\end{center}
\end{frame}

\begin{frame}{Qui est là ?}
	\begin{tikzpicture}
			\node[draw] (E1) at (-3,0) {Entité 1};
			\node[draw] (E2) at (3,0) {Entité 3};
			\node[draw] (N) at (0,1) {Entité 2};
			\visible<1>{
				\draw[->,>=latex] (E1) to[bend left] node[sloped,midway,above]{WHOS 1 + MEMB 1} (N);
				\draw[->,>=latex] (E2) to[bend left] (E1);
				\draw[->,>=latex] (N) to[bend left] (E2);
			}
			\visible<2>{
				\draw[->,>=latex] (E2) to[bend left] (E1);
				\draw[->,>=latex] (E1) to[bend left] (N);
				\draw[->,>=latex] (N) to[bend left] node[sloped,midway,above]{WHOS 1 + MEMB 1}  node[sloped,midway,below]{+ MEMB 2} (E2);
			}
			\visible<3>{
				\draw[->,>=latex] (E2) to[bend left] node[sloped,midway,above]{WHOS 1 + MEMB 1} node[sloped,midway,below]{+ MEMB 2 + MEMB 3} (E1);
				\draw[->,>=latex] (N) to[bend left] (E2);
				\draw[->,>=latex] (E1) to[bend left] (N);
			}
			\visible<4>{
				\draw[->,>=latex] (E1) to[bend left] node[sloped,midway,above]{MEMB 2 + MEMB 3}(N);
				\draw[->,>=latex] (E2) to[bend left] (E1);
				\draw[->,>=latex] (N) to[bend left] (E2);
			}
			\visible<5>{
				\draw[->,>=latex] (E1) to[bend left](N);
				\draw[->,>=latex] (E2) to[bend left] (E1);
				\draw[->,>=latex] (N) to[bend left] node[sloped,midway,above] {MEMB 3} (E2);
			}
	\end{tikzpicture}
	\begin{itemize}
		\item<1-5> Une entité 1 envoi \textbf{WHOS idm}
		\item<2-5> Les n autres entités reçoivent et renvoi WHOS idm
		\item<2-5> Elles envoient \textbf{MEMB idm id ip port} son IP et son port d'écoute
		\item<2-5> Elles reçoivent MEMB idm id ip port et le renvoient
		\item<4-5> Quand entité 1 reçoit WHOS idm , envoi son IP et son port d'écoute
		\item<2-5> Quand une entité reçoit MEMB son\_num\_port son\_IP => affichent
	\end{itemize}
\end{frame}

\subsubsection{Suppression entité}
\begin{frame}
	\begin{center}
		{\Huge Suppression entité}
	\end{center}
\end{frame}

\begin{frame}{Suppression entité}
	\begin{tikzpicture}
		\node[draw] (E1) at (-3,0) {Entité 1};
		\node[draw] (E2) at (3,0) {Entité 3};
		\node[draw] (N) at (0,1) {A supprimer};
		\visible<1>{
			\draw[->,>=latex] (E1) to[bend left] (N);
			\draw[->,>=latex] (N) to[bend left] node[sloped,midway,above] {GBYE} (E2);
			\draw[->,>=latex] (E2) to[bend left] (E1);
		}
		\visible<2>{
			\draw[->,>=latex] (E1) to[bend left] (N);
			\draw[->,>=latex] (N) to[bend left] (E2);
			\draw[->,>=latex] (E2) to[bend left] node[sloped,midway,above] {GBYE} (E1);
		}
		\visible<3>{
			\draw[->,>=latex] (E1) to[bend left] node[sloped,midway,above] {EYBG}  (N);
			\draw[->,>=latex] (N) to[bend left] (E2);
			\draw[->,>=latex] (E2) to[bend left] (E1);
			\draw[->,>=latex] (E1) -- (E2);
		}
		\visible<4>{
			\draw[->,>=latex] (E1) -- (E2);
			\draw[->,>=latex] (E2) to[bend left] (E1);
		}
	\end{tikzpicture}

	\begin{itemize}
		 \item<1-4> Entité envoi \textbf{GBYE idm ip port ip-succ port-succ} son identité et celle du suivant
		 \item<2-4> Entités suivantes reçoivent le message et le transmette
		 \item<3-4> Entité précédente reçoit le message, ne le retransmet pas (on sait que c'est la précédente car l'identité du message correspond à l'identité suivante de l'entité courante
		 \item<3-4> Ce dernier envoi \textbf{EYBG idm} et modifie ses attributs d'identité suivante avec les nouvelles
		 \item<4-4> Entité reçoit EYBG id, ne le retransmet pas, peut maintenant être supprimé
	\end{itemize}
\end{frame}

\subsubsection{Tester un anneau}
\begin{frame}
	\begin{center}
		{\Huge Tester un anneau}
	\end{center}
\end{frame}

\begin{frame}{Tester un anneau}
	\begin{tikzpicture}
		\node[draw] (E1) at (-3,0) {Entité 1};
		\node[draw] (E2) at (3,0) {Entité 3};
		\node[draw] (N) at (0,1) {Entité 2};
		\visible<1>{
			\draw[->,>=latex] (E1) to[bend left] node[sloped,midway,above] {TEST} (N);
			\draw[->,>=latex] (N) to[bend left] (E2);
			\draw[->,>=latex] (E2) to[bend left] (E1);
		}
		\visible<2>{
			\draw[->,>=latex] (E1) to[bend left] (N);
			\draw[->,>=latex] (N) to[bend left] node[sloped,midway,above] {TEST} (E2);
			\draw[->,>=latex] (E2) to[bend left] (E1);
		}
		\visible<3>{
			\draw[->,>=latex] (E1) to[bend left] (N);
			\draw[->,>=latex] (N) to[bend left] (E2);
			\draw[->,>=latex] (E2) to[bend left] node[sloped,midway,above] {TEST} (E1);
		}
		\visible<4>{
			\draw[->,>=latex] (E1) to[bend left](N);
			\draw[->,>=latex] (N) to[bend left] (E2);
			\draw[->,>=latex] (E2) to[bend left] (E1);
			
			\draw[->,>=latex,red] (E1) -- node[sloped,midway,above] {DOWN} (N);
			\draw[->,>=latex,red] (E1) -- node[sloped,midway,above] {DOWN} (E2);
			\draw[->,>=latex,red] (E1.north west) to [loop] node[sloped,midway,above] {DOWN} (E1);

		}
	\end{tikzpicture}
	\begin{itemize}
		\item<1-4> Entité 1 envoi \textbf{TEST idm ip-diff port-diff} son adresse Multi-diff
		\item<2-4> Quand une entité reçoit ce message elle le retransmet
		\item<3-4> Quand l'entité 1 reçoit le message alors l'anneau est correct
		\item<4-4> Si l'entité 1 ne reçoit pas le message au bout de n secondes, elle envoi un message \textbf{DOWN} sur son adresse de multi diffusion
		\item<4-4> Si une entité reçoit DOWN sur son adresse de multi diffusion, elle se termine
	\end{itemize}
\end{frame}

\subsubsection{Duplication}
\begin{frame}
	\begin{center}
		{\Huge La duplication}
	\end{center}
\end{frame}

\begin{frame}{La duplication :}
	\begin{tikzpicture}
			\node[draw] (E1) at (-3,0) {Entité 1};
			\node[draw] (D) at (2,0) {A dupliquer};
			\node[draw] (E2) at (-0.5,1) {Entité 2};
			\node[draw] (E4) at (5.8,0) {Entité 4};
			\visible<1-4>{
				\draw[->,>=latex] (E4.north east) to [loop] (E4);
			}
			\visible<1>{
				\draw[->,>=latex] (E4) -- node[sloped,midway,above]{connect} (D);
			}
			\visible<2>{
				\draw[->,>=latex] (D) -- node[sloped,midway,above]{WELC} (E4);
			}
			\visible<3>{
				\draw[->,>=latex] (E4) -- node[sloped,midway,above]{DUPL} (D);
			}
			\visible<4>{
				\draw[->,>=latex] (D) -- node[sloped,midway,above]{ACKD} (E4);
			}
			\visible<5-6>{
			 	\draw[->,>=latex] (E4) to[bend right] (D);
			}
			\visible<6>{
				\draw[->,>=latex] (D) to[bend right] (E4);

			}
			\draw[->,>=latex] (E1) to[bend left](E2);
			\draw[->,>=latex] (D) to[bend left] (E1);
			\draw[->,>=latex] (E2) to[bend left] (D);

	\end{tikzpicture}
	\begin{itemize}
		\item<1-6> Entité 4 se connecte à Entité 3
		\item<2-6> Entité 4 envoi \textbf{WELC ip port ip-diff port-diff \escape{n}} (adresse et port de l’entité 2, adresse et port de multi-diffusion)
		\item<2-6> Nouveau reçoit WELC ip port ip-diff port-diff \escape{n}, stocke
		\item<3-6> Nouveau envoi \textbf{DUPL ip port ip-diff port-diff\escape{n}} (information qu'il connaît dans ses attributs)
		\item<3-6> Entité 4 reçoit DUPL ip port ip-diff port-diff\escape{n}, stocke
		\item<4-6> Entité 4 envoi \textbf{ACKD port\escape{n}}
		\item<5-6> Entité 4 ferme la connexion
		\item<4-6> Nouveau reçoit ACKD port \escape{n}, stocke port
	\end{itemize}
\end{frame}

\subsubsection{Suppression messages}
\begin{frame}
	\begin{center}
		{\Huge La suppression des messages}
	\end{center}
\end{frame}

\begin{frame}{La suppression des messages :}
\framesubtitle{Algorithme et explication de comment on supprime pour éviter l'accumulation dans la liste et allongement du temps de recherche}
\begin{multicols}{2}
	\begin{tikzpicture}
		\node[draw=none,dashed,text width=2cm, text centered] (DT) at (-1,3.35) {};
		\node[draw=none] (H) at (-1,4.5) {};
		\visible<1->{
			\draw[->,>=latex] (H) -- node[left] {Recoi X}  node[right]{sauf SUPP} (DT);
		}
		\visible<2->{
			\node[draw,dashed,text width=2cm, text centered] (DT) at (-1,3) {Deja transmis};
		}
		\visible<3->{
			\node[draw, text width=2cm, text centered] (TM) at (0.2,1) {Transfert Message};
			\draw[->,>=latex] (DT) -- node[right] {False} (TM);
		}
		\visible<4->{
			\node[draw,dashed,text width=2cm, text centered] (D) at (-2.5,1) {Dupplicateur};
			\draw[->,>=latex] (DT) -- node[left] {True} (D);
		}
		\visible<5->{
			\node[draw, text width=2cm, text centered] (TCS) at (-1.3,-2) {Remove + Cree SUPP};
			\draw[->,>=latex] (D) -- node[right] {False} (TCS);
		}
		\visible<6->{
			\node[draw, text centered] (R) at (-3,-2) {Rien};
			\draw[->,>=latex] (D) -- node[left] {True} (R);
		}
		\draw (1.5,4.5) -- (1.5,-2.5);		
	\end{tikzpicture}
	
	\begin{tikzpicture}
		\visible<7->{
			\draw[->,>=latex] (H) -- node[left] {Recoi }  node[right]{SUPP} (DT);
		}
		\visible<8->{
			\node[draw,dashed,text width=2cm, text centered] (DT) at (-1,3) {Deja transmis};
		}
		\visible<9->{
			\node[draw, text centered] (TM) at (0.2,1) {Rien};
			\draw[->,>=latex] (DT) -- node[right] {False} (TM);
		}
		\visible<10->{
			\node[draw,dashed,text width=2cm, text centered] (D) at (-2.5,1) {Dupplicateur};
			\draw[->,>=latex] (DT) -- node[left] {True} (D);
		}
		\visible<11->{
			\node[draw, text width=1.5cm, text centered] (TCS) at (-0.9,-1.6) {Remove idmA + Transfert SUPP};
			\draw[->,>=latex] (D) -- node[right] {False} (TCS);
		}
		\visible<12>{
			\node[draw, text width=2cm, text centered] (R) at (-3,-2) {Transfert Message};
			\draw[->,>=latex] (D) -- node[left] {True} (R);
		}
	\end{tikzpicture}
 \end{multicols}
\end{frame}

\begin{frame}{La suppression des messages :}
\framesubtitle{Un exemple}
	\begin{tikzpicture}
		\visible<1>{
			\node[draw, text width=2.2cm, text centered] (E1) at (-3,0) {Entité 1 Transmis : 11};
		}
		\visible<1-2>{
			\node[draw, text width=2.2cm, text centered] (E2) at (-0.5,1.5) {Entité 2 Transmis : 11};
		}
		\node[draw, text width=2.2cm, text centered] (E3) at (2,0) {Entité 3 Transmis : 11};
		\visible<1-4>{
			\node[draw, text width=2.2cm, text centered] (E4) at (5,0) {Entité 4 Transmis : 11};
		}
		\visible<1>{
			\draw[->,>=latex] (E1) to[bend left] (E2);
			\draw[->,>=latex] (E2) to[bend left] (E3);
			\draw[->,>=latex] (E3) to[bend left] node[sloped,midway,above]{X 11} (E1);
			\draw[->,>=latex] (E3) to[bend right] (E4);
			\draw[->,>=latex] (E4) to[bend right] (E3);
		}
		\visible<2-7>{
			\node[draw, text width=2.2cm, text centered] (E1) at (-3,0) {Entité 1};
		}
		\visible<2>{
			\draw[->,>=latex] (E1) to[bend left] node[sloped,midway,above]{SUPP} (E2);
			\draw[->,>=latex] (E2) to[bend left] (E3);
			\draw[->,>=latex] (E3) to[bend left] (E1);
			\draw[->,>=latex] (E3) to[bend right] (E4);
			\draw[->,>=latex] (E4) to[bend right] (E3);
		}
		\visible<3-7>{
			\node[draw, text width=2.2cm, text centered] (E2) at (-0.5,1.5) {Entité 2};
		}
		\visible<3>{
			\draw[->,>=latex] (E1) to[bend left] (E2);
			\draw[->,>=latex] (E2) to[bend left] node[sloped,midway,above]{SUPP} (E3);
			\draw[->,>=latex] (E3) to[bend left] (E1);
			\draw[->,>=latex] (E3) to[bend right] (E4);
			\draw[->,>=latex] (E4) to[bend right] (E3);
		}
		\visible<4>{
			\draw[->,>=latex] (E1) to[bend left] (E2);
			\draw[->,>=latex] (E2) to[bend left] (E3);
			\draw[->,>=latex] (E3) to[bend left] node[sloped,midway,above]{SUPP} (E1);
			\draw[->,>=latex] (E3) to[bend right] node[sloped,midway,below]{SUPP} (E4);
			\draw[->,>=latex] (E4) to[bend right] (E3);
		}
		\visible<5-7>{
			\node[draw, text width=2.2cm, text centered] (E4) at (5,0) {Entité 4 };
		}
		\visible<5>{
			\draw[->,>=latex] (E1) to[bend left]  node[sloped,midway,above]{rien}(E2);
			\draw[->,>=latex] (E2) to[bend left] (E3);
			\draw[->,>=latex] (E3) to[bend left] (E1);
			\draw[->,>=latex] (E3) to[bend right] (E4);
			\draw[->,>=latex] (E4) to[bend right] node[sloped,midway,above]{SUPP} (E3);
		}
		\visible<6>{
			\draw[->,>=latex] (E1) to[bend left] (E2);
			\draw[->,>=latex] (E2) to[bend left] (E3);
			\draw[->,>=latex] (E3) to[bend left] node[sloped,midway,above]{SUPP} (E1);
			\draw[->,>=latex] (E3) to[bend right] node[sloped,midway,below]{SUPP} (E4);
			\draw[->,>=latex] (E4) to[bend right] (E3);
		}
		\visible<7>{
			\draw[->,>=latex] (E1) to[bend left] node[sloped,midway,above]{rien}(E2);
			\draw[->,>=latex] (E2) to[bend left] (E3);
			\draw[->,>=latex] (E3) to[bend left] (E1);
			\draw[->,>=latex] (E3) to[bend right] (E4);
			\draw[->,>=latex] (E4) to[bend right] node[sloped,midway,above]{rien} (E3);
		}
	\end{tikzpicture}
	\begin{itemize}
		\item<1-7> Entité 1 reçoit un message qu'elle a déjà envoyé
		\item<2-7> Entité 1 non dupl, supprime idmA, envoi SUPP
		\item<3-7> Entité 2 non dupl reçoit SUPP, contient puis supprime idmA, transmet SUPP
		\item<4-7> Entité 3 dupl reçoit SUPP, contient idmA, transmet SUPP
		\item<5-7> E1 reçoit SUPP, non contenu ne transmet pas et E4 -> E2
		\item<6-7> Entité 3 => Entité 3 
		\item<7-7> Entité 1 et 4 => Entité 1
	\end{itemize}
\end{frame}

\subsection{Messages d'application}
\begin{frame}
	\begin{center}
		{\Huge Messages d'application}
	\end{center}
\end{frame}

\subsubsection{Diffusion}
\begin{frame}
	\begin{center}
		{\Huge Diffusion de messages à tout le monde}
	\end{center}
\end{frame}

\begin{frame}{Diffusion de messages à tout le monde :}
	\begin{tikzpicture}
		\node[draw] (E1) at (-3,0) {Entité 1};
		\node[draw] (E2) at (3,0) {Entité 3};
		\node[draw] (N) at (0,1) {Entité 2};
		\visible<1>{
			\draw[->,>=latex] (E1) to[bend left] node[sloped,midway,above]{DIFF} (N);
			\draw[->,>=latex] (E2) to[bend left] (E1);
			\draw[->,>=latex] (N) to[bend left] (E2);
		}
		\visible<2>{
			\draw[->,>=latex] (E2) to[bend left] (E1);
			\draw[->,>=latex] (E1) to[bend left] (N);
			\draw[->,>=latex] (N) to[bend left] node[sloped,midway,above]{DIFF} (E2);
		}
		\visible<3>{
			\draw[->,>=latex] (E2) to[bend left] node[sloped,midway,above]{DIFF} (E1);
			\draw[->,>=latex] (N) to[bend left] (E2);
			\draw[->,>=latex] (E1) to[bend left] (N);
		}
	\end{tikzpicture}
	\begin{itemize}
		 \item<1-3> Entité 1 envoi \textbf{APPL idm DIFF\#\#\#\# size-mess mess}
		 \item<2-3> Quand un entité reçoit DIFF, elle le retransmet
		 \item<3-3> Quand entité 1 reçoit DIFF elle ne le retransmet pas
	\end{itemize}
\end{frame}

\subsubsection{Transfert}
\begin{frame}
	\begin{center}
		{\Huge Transfert de fichiers}
	\end{center}
\end{frame}

\begin{frame}{Transfert de fichiers :}
	\framesubtitle{Cas où le fichier n'est pas présent}
	\begin{tikzpicture}
		\node[draw] (E1) at (-3,0) {Entité 1};
		\node[draw] (E2) at (3,0) {Entité 3};
		\node[draw] (N) at (0,1) {Entité 2};
		\visible<1>{
			\draw[->,>=latex] (E1) to[bend left] node[sloped,midway,above]{TRANS} (N);
			\draw[->,>=latex] (E2) to[bend left] (E1);
			\draw[->,>=latex] (N) to[bend left] (E2);
		}
		\visible<2>{
			\draw[->,>=latex] (E2) to[bend left] (E1);
			\draw[->,>=latex] (E1) to[bend left] (N);
			\draw[->,>=latex] (N) to[bend left] node[sloped,midway,above]{TRANS} (E2);
		}
		\visible<3>{
			\draw[->,>=latex] (E2) to[bend left] node[sloped,midway,above]{TRANS}  (E1);
			\draw[->,>=latex] (N) to[bend left] (E2);
			\draw[->,>=latex] (E1) to[bend left] (N);
		}
	\end{tikzpicture}
	\begin{itemize}
		\item<1-3> Une entité 1 envoi \textbf{TRANS}
		\item<2-3> Une entité reçoit TRANS si elle n'a pas le fichier dans son répertoire courant, elle retransmet
		\item<3-3> Entité 1 reçoit TRANS alors elle sait que personne a le fichier
	\end{itemize}
\end{frame}

\begin{frame}{Transfert de fichiers :}
	\framesubtitle{Cas où le fichier est présent}
	\begin{tikzpicture}
		\node[draw] (E1) at (-3,0) {Entité 1};
		\node[draw] (E2) at (3,0) {Entité 3};
		\node[draw] (N) at (0,1) {Entité 2};
		\visible<1>{
			\draw[->,>=latex] (E1) to[bend left] node[sloped,midway,above]{TRANS} (N);
			\draw[->,>=latex] (E2) to[bend left] (E1);
			\draw[->,>=latex] (N) to[bend left] (E2);
		}
		\visible<2>{
			\draw[->,>=latex] (E1) to[bend left] (N);
			\draw[->,>=latex] (E2) to[bend left] (E1);
			\draw[->,>=latex] (N) to[bend left] node[sloped,midway,above]{ROK + SEN} (E2);
		}
		\visible<3>{
			\draw[->,>=latex] (E1) to[bend left](N);
			\draw[->,>=latex] (E2) to[bend left] node[sloped,midway,above] {ROK + SEN} (E1);
			\draw[->,>=latex] (N) to[bend left] (E2);
		}
		
		\visible<4>{
			\draw[->,>=latex] (E1) to[bend left] node[sloped,midway,above] {BEUG => REQ}(N);
			\draw[->,>=latex] (E2) to[bend left] (E1);
			\draw[->,>=latex] (N) to[bend left] (E2);
		}
	\end{tikzpicture}
	\begin{itemize}
		\item<1-4> Une entité 1 envoi \textbf{TRANS}
		\item<2-4> Entité ayant file recevant TRANS => ne renvoi pas REQ
		\item<2-4> Elle envoi d'abord \textbf{APPL idm TRANS\#\#\# ROK id-trans size-nom nom-fichier nummess}
		\item<2-4> Puis elle envoi n \textbf{APPL idm TRANS\#\#\# SEN id-trans no-mess size-content content}
		\item<3-4> Entité voulant pas file recevant SEN ou ROK renvoi mssg
		\item<4-4> Entité 1 reçoit SEN et ROK crée le file, si les messages ne sont pas reçus dans l'ordre, renvoi demande du file
	\end{itemize}
\end{frame}

\end{document}

