\documentclass{beamer}

\usepackage[utf8]{inputenc}
\usepackage[french]{babel}
\usepackage[T1]{fontenc}
\usepackage{lmodern}
\usepackage{tikz}
\usepackage{multicol}

\newcommand*{\escape}[1]{\texttt{\textbackslash#1}}

\usetheme{PaloAlto}

\title{Protocole RINGO}
\subtitle{Rapport du Projet}
\author[]{ELBEZ Samuel 21200353, \\ JACQUETTE Pierrick 21305551}
\date{Avril 2016}
\institute[L3 S6 -- Informatique]{Université Paris 7 Diderot}

\begin{document}

\begin{frame}
	\titlepage
\end{frame}

\begin{frame}
	\frametitle{Sommaire}
	\tableofcontents	
\end{frame}

\section{Une entité}
\begin{frame}{Une entité}
	 \textbf{Les caractéristiques d'une entité}
	 \begin{itemize}
		 \item<2-10> Un identifiant (8 caractères et unique)
 		 \item<3-10> Un port d'écoute pour recevoir en UDP (< 9999)
		 \item<4-10> Un port TCP
		 \item<5-10> Une adresse IP où l'on envoi
		 \item<6-10> Un port d'écoute UDP où l'on envoi (< 9999)
		 \item<7-10> Un tableau d'adresse IPV4 multi-diffusion (panne du réseau)
		 \item<8-10> Un tableau de port de multi-diffusion (< 9999) 
		 \item<9-10> Un boolean pour savoir si c'est un duplicateur
		 \item<10-10> Un tableau de liste de message déjà retransmis
	\end{itemize}
\end{frame}

\section{Les ports }
\begin{frame}{L'utilisation des différents ports}
	 \begin{itemize}
		 \item<1-2> \textbf{UDP} écoute et transmet les messages
 		 \item<2-2> \textbf{TCP} permet l'insertion ou duplication d'un anneau
	\end{itemize}
\end{frame}

\section{Les conventions }
\begin{frame}{Les conventions}
	 \begin{itemize}
		 \item<1-5> La syntaxe des messages
		 \item<2-5> Un message a une taille maximale de 512 octets
 		 \item<3-5> Une entité recevant un message déjà transmis, on ne le retransmet pas
 		 \item<4-5> Si on reçoit un message mal formé, on ne le retransmet pas
 		 \item<5-5> Une entité reçoit un message d'application si elle supporte déclenche une action sinon transmet
	\end{itemize}
\end{frame}

\section{Les messages}
\begin{frame}{Les différents types de messages}
	\textbf{Les messages d'application}
	\begin{itemize}
		 \item<1-9> APPL idm id-app message-app
	\end{itemize}
	\textbf{Les messages du protocole}
	\begin{itemize}
		 \item<2-9> \textbf{L'insertion : } WELC ip port ip-diff port-diff \escape{n}
% 		 \item<3-8> \textbf{f} ff
%		 \item<4-9> \textbf{}
%		 \item<5-9> \textbf{}
%		 \item<6-9> \textbf{}
%		 \item<7-9> \textbf{} 
%		 \item<8-9> \textbf{} 
%		 \item<9-9> \textbf{} 
	\end{itemize}
\end{frame}


\begin{frame}{Les applications}
	\begin{itemize}
		 \item<1-2> \textbf{Diffusion de messages à tout le monde :} APPL idm DIFF\#\#\#\# size-mess mess
 		 \item<2-2> \textbf{Transfert de fichiers :} 
	\end{itemize}
\end{frame}

\begin{frame}{L'insertion}
	\begin{multicols}{2}
   		\visible<1-7>{
			\begin{tikzpicture}
				\node[draw] (E1) at (-1.5,0) {Entité 1};
				\node[draw] (E2) at (1.5,0) {Entité 2};
				\node[draw] (N) at (0,1) {Nouveau};
				\node[draw=none] (d1) at (0,-0.1) {};
				\draw[->] (E1) -- (E2);
				\draw[->] (N) -- (d1);
			\end{tikzpicture}
		}
		\visible<2-7>{
			\begin{tikzpicture}
				\node[draw] (E1) at (2,0) {Entité 1};
				\node[draw] (E2) at (5,0) {Entité 2};
				\node[draw] (N) at (3.5,1) {Nouveau};
				\draw[->,>=latex] (E1) to[bend left] (N);
				\draw[->,>=latex] (N) to[bend left] (E2);
			\end{tikzpicture}
		}
	\end{multicols}
	\begin{itemize}
		\item<3-7> Nouveau se connecte à Entité 1
		\item<4-7> Entité 1 envoi \textbf{WELC ip port ip-diff port-diff \escape{n}} (information qu'il connaît dans ses attributs : adresse et port de l’entité 2, adresse et port de multi-diffusion)
		\item<4-> Nouveau reçoit WELC ip port ip-diff port-diff \escape{n}, stocke
		\item<5-7> Nouveau envoi \textbf{NEWC ip port \escape{n}} (information qu'il connaît dans ses attributs : adresse et port de Nouveau)
		\item<5-7> Entité 1 reçoit NEWC ip port \escape{n}, stocke
		\item<6-7> Entité 1 envoi \textbf{ACKC \escape{n}}
		\item<7-7> Entité 1 ferme la connexion
		\item<6-7> Nouveau reçoit ACKC \escape{n} et ne le transfert pas
	\end{itemize}
\end{frame}

\end{document}

