\documentclass{beamer}

\usepackage[utf8]{inputenc}
\usepackage[french]{babel}
\usepackage[T1]{fontenc}
\usepackage{lmodern}
\usepackage{tikz}
\usepackage{multicol}

\newcommand*{\escape}[1]{\texttt{\textbackslash#1}}

\usetheme{PaloAlto}

\title{Protocole RINGO}
\subtitle{Rapport du Projet}
\author[]{ELBEZ Samuel 21200353, \\ JACQUETTE Pierrick 21305551}
\date{Avril 2016}
\institute[L3 S6 -- Informatique]{Université Paris 7 Diderot}

\begin{document}

\begin{frame}
	\titlepage
\end{frame}

\begin{frame}
	\frametitle{Sommaire}
	\tableofcontents	
\end{frame}

\section{Une entité}
\begin{frame}{Une entité}
	 \textbf{Les caractéristiques d'une entité}
	 \begin{itemize}
		 \item<2-10> Un identifiant (8 caractères et unique)
 		 \item<3-10> Une chaîne de caractères pour le port de reception en UDP (< 9999)
		 \item<4-10> Une chaîne de caractères pour le port d'envoi en UDP (< 9999)
		 \item<5-10> Une chaîne de caractères pour l'adresse IP où l'on envoi
		 \item<6-10> Une chaîne de caractères pour le port TCP
		 \item<7-10> Un tableau de chaîne de caractères pour l'adresse IPV4 multi-diffusion (panne du réseau)
		 \item<8-10> Un tableau de chaîne de caractères pour le port de multi-diffusion (< 9999) 
		 \item<9-10> Un boolean pour savoir si c'est un duplicateur
		 \item<10-10> Un tableau de liste de chaîne de caractères de l'identifiant du message transmis
	\end{itemize}
\end{frame}

\section{Les ports}
\begin{frame}{L'utilisation des différents ports}
	 \begin{itemize}
		 \item<1-2> \textbf{UDP} écoute et transmet les messages
 		 \item<2-2> \textbf{TCP} permet l'insertion ou duplication d'un anneau
	\end{itemize}
\end{frame}

\section{Les conventions}
\begin{frame}{Les conventions}
	 \begin{itemize}
		 \item<1-5> La syntaxe des messages
		 \item<2-5> Un message a une taille maximale de 512 octets
 		 \item<3-5> Une entité recevant un message déjà transmis, on ne le retransmet pas
 		 \item<4-5> Si on reçoit un message mal formé, on ne le retransmet pas
 		 \item<5-5> Une entité reçoit un message d'application si elle supporte déclenche une action sinon transmet
	\end{itemize}
\end{frame}

\section{La spécification}
\begin{frame}{La spécification des messages}
	 \begin{itemize}
		 \item<1-11> Les 4 premiers octets : \textbf{WELC, NEWC, ACKC, APPL, WHOS, MEMB, GBYE, EYBG, TEST, DOWN, DUPL, ACKD}
		 \item<2-11> \textbf{ip, ip-diff et ip-succ :} 15 oct, chaîne de caractères de l'adresse IPV4
 		 \item<3-11> \textbf{port, port-diff et port-succ :} 4 oct, chaîne de caractères du numéro de port
 		 \item<4-11> \textbf{idm et id-trans :} 8 oct unique
 		 \item<5-11> \textbf{id-app :} 8 oct, chaîne de caractères de l'identité de l'app
 		 \item<6-11> \textbf{id :} 8 oct, chaîne de caractères
 		 \item<7-11> \textbf{size-mess :} 3 oct, chaîne de caractères 
 		 \item<8-11> \textbf{size-nom :} 2 oct, chaîne de caractères
 		 \item<9-11> \textbf{num-mess et no-mess :} 8 oct, little endian
 		 \item<10-11> \textbf{size-content :} 3 oct, chaîne de caractères
	\end{itemize}
	\visible<11-11>{
		size-mess, size-nom et size-content complété par des 0 au début
	}
\end{frame}

\section{Les messages}
\begin{frame}{Les différents types de messages}
	\textbf{Les messages d'application}
	\begin{itemize}
		 \item<2-9> APPL idm id-app message-app
	\end{itemize}
	\textbf{Les messages du protocole}
	\begin{itemize}
		 \item<3-9> \textbf{L'insertion :} WELC ip port ip-diff port-diff \escape{n}
 		 \item<4-9> \textbf{Qui :} WHOS idm
		 \item<5-9> \textbf{Suppression :} GBYE idm ip port ip-succ port-succ
		 \item<6-9> \textbf{Tester un anneau :} TEST idm ip-diff port-diff
%		 \item<7-9> \textbf{} 
%		 \item<8-9> \textbf{} 
%		 \item<9-9> \textbf{} 
	\end{itemize}
\end{frame}


\begin{frame}{Les applications}
	\begin{itemize}
		 \item<1-2> \textbf{Diffusion de messages à tout le monde :} APPL idm DIFF\#\#\#\# size-mess mess
 		 \item<2-2> \textbf{Transfert de fichiers :} A COMPLETER
	\end{itemize}
\end{frame}

\begin{frame}{L'insertion}
	\begin{tikzpicture}
			\node[draw] (E1) at (-3,0) {Entité 1};
			\node[draw] (E2) at (3,0) {Entité 2};
			\node[draw] (N) at (0,1) {Nouveau};
			\node[draw=none] (d1) at (0,-0.1) {};
			\visible<1>{
				\draw[->,>=latex] (N) -- (d1);
			}
			\visible<1-5>{
				\draw[->,>=latex] (E1) -- (E2);
			}
			\visible<2>{
				\draw[->,>=latex] (E1) to[bend left] node[sloped,midway,above]{connect} (N);
			}
			\visible<3>{
				\draw[->,>=latex] (E1) to[bend left] node[sloped,midway,above]{WELC} (N);
			}
			\visible<4>{
				\draw[->,>=latex] (N) to[bend right] node[sloped,midway,above]{NEWC} (E1);
			}
			\visible<5>{
				\draw[->,>=latex] (E1) to[bend left] node[sloped,midway,above]{ACKC} (N);
			}
			\visible<6>{
				\draw[->,>=latex] (E1) to[bend left](N);
				\draw[->,>=latex] (N) to[bend left] (E2);
			}
			\draw[->,>=latex] (E2) to[bend left] (E1);

	\end{tikzpicture}
	\begin{itemize}
		\item<2-6> Nouveau se connecte à Entité 1
		\item<3-6> Entité 1 envoi \textbf{WELC ip port ip-diff port-diff \escape{n}} (adresse et port de l’entité 2, adresse et port de multi-diffusion)
		\item<3-6> Nouveau reçoit WELC ip port ip-diff port-diff \escape{n}, stocke
		\item<4-6> Nouveau envoi \textbf{NEWC ip port \escape{n}} (information qu'il connaît dans ses attributs : adresse et port de Nouveau)
		\item<4-6> Entité 1 reçoit NEWC ip port \escape{n}, stocke
		\item<5-6> Entité 1 envoi \textbf{ACKC \escape{n}}
		\item<6-6> Entité 1 ferme la connexion
		\item<5-6> Nouveau reçoit ACKC \escape{n} et ne le transfert pas
	\end{itemize}
\end{frame}

\begin{frame}{Qui est là ?}
	\begin{itemize}
		\item<1-5> Une entité 1 envoi \textbf{WHOS idm}
		\item<2-5> Les n autres entités reçoivent et renvoi WHOS idm
		\item<2-5> Elles envoient \textbf{MEMB idm id ip port} son IP et son port d'écoute
		\item<3-5> Elles reçoivent MEMB idm id ip port et le renvoient
		\item<4-5> Quand entité 1 reçoit WHOS idm , envoi son IP et son port d'écoute
		\item<5-5> Quand chaque entité reçoit MEMB avec son numéro de port et son IP il les affichent
	\end{itemize}
\end{frame}

\begin{frame}{Suppression}
	\begin{multicols}{2}
   		\visible<1-6>{
			\begin{tikzpicture}
				\node[draw] (E1) at (2,0) {Entité 1};
				\node[draw] (E2) at (5,0) {Entité 2};
				\node[draw] (N) at (3.5,1) {A Supprimer};
				\draw[->,>=latex] (E1) to[bend left] (N);
				\draw[->,>=latex] (N) to[bend left] (E2);
				\draw[->,>=latex] (E2) to[bend left] (E1);
			\end{tikzpicture}
		}
		\visible<2-6>{
			\begin{tikzpicture}
				\node[draw] (E1) at (2,0) {Entité 1};
				\node[draw] (E2) at (5,0) {Entité 2};
				\draw[->,>=latex] (E1) to[bend left] (E2);
				\draw[->,>=latex] (E2) to[bend left] (E1);
			\end{tikzpicture}
		}
	\end{multicols}
	\begin{itemize}
		 \item<3-6> Entité envoi \textbf{GBYE idm ip port ip-succ port-succ} son identité et celle du suivant
		 \item<4-6> Entités suivantes reçoivent le message et le transmette
		 \item<4-6> Entité précédente reçoit le message, ne le retransmet pas (on sait que c'est la précédente car l'identité du message correspond à l'identité suivante de l'entité courante
		 \item<5-6> Ce dernier envoi \textbf{EYBG idm} et modifie ses attributs d'identité suivante avec les nouvelles
		 \item<6-6> Entité reçoit EYBG id, ne le retransmet pas, peut maintenant être supprimé
	\end{itemize}
\end{frame}

\begin{frame}{Tester un anneau}
	\begin{itemize}
		\item
		\item
	\end{itemize}
\end{frame}

\end{document}

